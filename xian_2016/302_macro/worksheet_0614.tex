\documentclass{article}

\usepackage{amsmath}

\begin{document}

\title{Intermediate Macro In-Class Problems \\ \large Exploring Romer Model}

\date{June 14, 2016}

\maketitle

Today we will explore the mechanisms of the simply Romer model by exploring how economies described by this model would react to exogenous changes.

First, recall this summary of the Romer model:

\begin{align*}
\text{Production} &: Y_t = A_t L^y_t \\
\text{Technological Dynamics} &: \Delta A_{t+1} = \bar{z} A_t L^a_t \\
\text{Resource Constraint} &: L^y_t + L^a_t = \bar{L} \\
\text{Labor Allocation Rule} &: L^a_t = \bar{l} \bar{L}
\end{align*}

And the model solution:

\begin{align*}
L^a_t &= \bar{l} \bar{L} \\
L^y_t &= \left(1 - \bar{l} \right) \bar{L} \\
A_t &= \left(1 + \bar{z} \bar{l} \bar{L} \right)^t A_0 \\
Y_t &= \left(1 - \bar{l} \right) \left(1 + \bar{z} \bar{l} \bar{L} \right)^t A_0 \bar{L}
\end{align*}

\section{Labor Force}

Analyze the effects of a government policy encouraging immigration in the context of the Romer model.

Remember, any proper analysis of policy in a macro model will evaluate the impact of the policy on every aspect of the model -- typically, the model parameters, but occasionally also the functional forms (themselves a higher-dimensional parameter).

\subsection*{Identifying Model Parameters}

What are the parameters of the Romer model? What are the quantities that are taken as given, and which are allowed to change as a result of agents' behavior?

\subsection*{Evaluating the Policy}

What is the potential effect of the immigration policy on each of the parameters you identified above?

\section{Reallocation of Labor}

Analyze the effects of a government policy which subsidizes research in the context of the Romer model.

What is the potential effect of the immigration policy on each of the parameters you identified above?

\section{Textbook Exercises}

\subsection*{6.1}

Explain whether the following goods are rivalrous or nonrivalrous:

\begin{enumerate}
\item Beethoven's Fifth Symphony
\item An iPod
\item Monet's painting \textit{Water Lilies}
\item The method of public key cryptography (RSA)
\item Fish in the ocean
\end{enumerate}

\subsection*{6.2}

Suppose a new piece of computer software -- say a word processor with perfect speech recognition -- can be created for a onetime cost of \$100 million. Suppose that once it's created, copies of the software can be distributed at a cost of \$1 each.

\begin{enumerate}
\item If $Y$ denotes the number of copies of the computer program produced and $X$ denotes the amount spent on production, what is the production function; that is, the relation between $Y$ and $X$?
\item Make a graph of this production function. Does it exhibit increasing returns? Why or why not?
\item Suppose the firm charges a price equal to marginal cost (\$1) and sells a million copies of hte software. What are its profits?
\item Suppose the firm charges a price of \$20. How many copies does it have to sell in order to break even? What if the price is \$100 per copy?
\item Why does the scale of the market -- the number of copies the firm could sell -- matter?
\end{enumerate}

\subsection*{6.8: A variation on the Romer model}

Consider the following variation:

\begin{align*}
Y_t &= A_t^{\frac12} L^y_T \\
\Delta A_{t+1} &= \bar{z} A_t L^a_t \\
L^y_t + L^a_t &= \bar{L} \\
L^a_t &= \bar{l} \bar{L}
\end{align*}

There is only a single difference: we've changed the exponent on $A_t$ in the production of the output good so that there is now a diminishing marginal product to ideas in that sector.

\begin{enumerate}
\item Provide an economic interpretation for each equation.
\item What is the growth rate of knowledge in this economy?
\item What is the growth rate of output per person in this economy?
\item Solve for the level of output per person at each point in time.
\end{enumerate}

\end{document}