\documentclass{article}

\usepackage{xcolor}

\begin{document}

\title{Intermediate Macro HW 2}

\maketitle

\section{Sino-Solow}

\begin{itemize}
\item In what ways does the Solow model do a good job of explaining the evolution of the Chinese economy in the past century?

\item In what ways does the Solow model fall short as an explanation of the evolution of China in the past century?

\color{red} 

Open-ended.

\color{black} 

\end{itemize}

\section{Textbook Exercises, Chapter 4}

Complete the following exercises from the textbook: 4.3, 4.8.

\color{red} 

\subsection*{4.3}

\subsubsection*{(a)}

In the production model, wages are pinned down by $w = MPL = \frac13 \bar{A} \left( \frac{K}{L} \right)^{\frac23}$; in turn, since the entire stock of capital and labor are consumed in this model, we have

\[ w =  \frac13 \bar{A} \left( \frac{\bar{K}}{\bar{L}} \right)^{\frac23} \]

The black death, as regards this model, effected a major decrease in the stock of labor, $\bar{L}$. Since $\frac{\partial w}{\partial \bar{L}} < 0$, a decrease in $\bar{L}$ means an increase in $w$.

Intuitively, since wages are given by the marginal product of labor, which is diminishing, less labor means higher marginal productivity. 

\subsubsection*{(b)}

The ratio of wages after ($w^{'}$) to before ($w$) the Black Death is:

\[ \frac{w^{'}}{w} = \frac{MPL^{'}}{MPL} = \frac{\frac13 \bar{A} \left( \frac{\bar{K}}{\bar{L}} \right)^{\frac23}}{\frac13 \bar{A} \left( \frac{\bar{K}}{\bar{L}^{'}} \right)^{\frac23}} = \left( \frac{\bar{L}}{\bar{L}^{'}} \right)^{\frac23}\]

Since $\bar{L}^{'} = \frac23 \bar{L}$, $\frac{w^{'}}{w} = \left( \frac32 \right)^{\frac23} \approx 1.31$, i.e., a 31\% increase.

\subsection*{4.8}

Open ended.

\color{black} 

\section{Textbook Exercises, Chapter 5}

Complete the following exercises from the textbook: 5.2, 5.5, 5.7

\color{red} 

\subsection*{5.2}

\subsubsection*{(a)}

The likely result is a rapid uptick in output, followed by a slowdown as the new steady state (which has a higher output level) is approached. The diagram will feature the depreciation curve unchanged (unless justification is given for why this should shift), and the investment curve shifted upwards (since it is proportional to $\bar{A}$).

This results in the economy being temporarily below the new steady state level of capital, so net investment turns positive until the new steady state is reached.

\subsubsection*{(b)}

See above. Output per capita in Solow is given by $y^{*} = \bar{A}^{\frac32} \left( \frac{\bar{s}}{\delta} \right)^{\frac12}$. So returns to $\bar{A}$ are convex -- output per person will grow by a higher fraction than does $\bar{A}$.

\subsubsection*{(c)}

The growth rate of output initially should be high; regardless, the growth rate eventually tempers towards zero as the steady state is neared.

\subsubsection*{(d)}

Open ended.

\end{document}