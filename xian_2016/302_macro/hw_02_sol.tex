\documentclass{article}

\usepackage{xcolor}

\begin{document}

\title{Intermediate Macro HW 2}

\maketitle

\section{Sino-Solow}

\begin{itemize}
\item In what ways does the Solow model do a good job of explaining the evolution of the Chinese economy in the past century?

\item In what ways does the Solow model fall short as an explanation of the evolution of China in the past century?

\color{red} 

Open-ended.

\color{black} 

\end{itemize}

\section{Textbook Exercises, Chapter 4}

Complete the following exercises from the textbook: 4.3, 4.8.

\color{red} 

\subsection*{4.3}

\subsubsection*{(a)}

In the production model, wages are pinned down by $w = MPL = \frac13 \bar{A} \left( \frac{K}{L} \right)^{\frac23}$; in turn, since the entire stock of capital and labor are consumed in this model, we have

\[ w =  \frac13 \bar{A} \left( \frac{\bar{K}}{\bar{L}} \right)^{\frac23} \]

The black death, as regards this model, effected a major decrease in the stock of labor, $\bar{L}$. Since $\frac{\partial w}{\partial \bar{L}} < 0$, a decrease in $\bar{L}$ means an increase in $w$.

Intuitively, since wages are given by the marginal product of labor, which is diminishing, less labor means higher marginal productivity. 

\subsubsection*{(b)}

The ratio of wages after ($w^{'}$) to before ($w$) the Black Death is:

\[ \frac{w^{'}}{w} = \frac{MPL^{'}}{MPL} = \frac{\frac13 \bar{A} \left( \frac{\bar{K}}{\bar{L}} \right)^{\frac23}}{\frac13 \bar{A} \left( \frac{\bar{K}}{\bar{L}^{'}} \right)^{\frac23}} = \left( \frac{\bar{L}}{\bar{L}^{'}} \right)^{\frac23}\]

Since $\bar{L}^{'} = \frac23 \bar{L}$, $\frac{w^{'}}{w} = \left( \frac32 \right)^{\frac23} \approx 1.31$, i.e., a 31\% increase.

\subsection*{4.8}

Open ended.

\color{black} 

\section{Textbook Exercises, Chapter 5}

Complete the following exercises from the textbook: 5.2, 5.5, 5.7

\color{red} 

\subsection*{5.2}

\subsubsection*{(a)}

The likely result is a rapid uptick in output, followed by a slowdown as the new steady state (which has a higher output level) is approached. The diagram will feature the depreciation curve unchanged (unless justification is given for why this should shift), and the investment curve shifted upwards (since it is proportional to $\bar{A}$).

This results in the economy being temporarily below the new steady state level of capital, so net investment turns positive until the new steady state is reached.

\subsubsection*{(b)}

See above. Output per capita in Solow is given by $y^{*} = \bar{A}^{\frac32} \left( \frac{\bar{s}}{\delta} \right)^{\frac12}$. So returns to $\bar{A}$ are convex -- output per person will grow by a higher fraction than does $\bar{A}$.

\subsubsection*{(c)}

The growth rate of output initially should be high; regardless, the growth rate eventually tempers towards zero as the steady state is neared.

\subsubsection*{(d)}

Open ended.

\subsection*{5.5}

\subsubsection*{(a)}

This basically jump-starts the economy to get much closer to its eventualy steady-state capital level. But since the fundamentals of the economy are unchanged (i.e., the parameters $\bar{A}$, $\bar{s}$, and $\delta$), the steady state of the economy is still the same.

Consumption per person in Solow is given by $c_t = \frac{C_t}{\bar{L}} = \frac{\left(1-\bar{s}\right) Y_t}{\bar{L}} = \left(1-\bar{s}\right) \bar{A} \left( \frac{K_t}{\bar{L}} \right)^{\frac13}$, so the ratio of output per person before and after the gift is:

\[ \frac{c_{t+1}}{c_t} = \frac{\left(1-\bar{s}\right) \bar{A} \left( \frac{K_{t+1}}{\bar{L}} \right)^{\frac13}}{\left(1-\bar{s}\right) \bar{A} \left( \frac{K_t}{\bar{L}} \right)^{\frac13}} = \left( \frac{K_{t+1}}{K_t} \right)^{\frac13} = \left( \frac43 \right)^{\frac13} \approx 1.10 \]

So, a 10\% increase in per-person consumption.

\subsubsection*{(b)}

Again, the long-run fundamentals of the economy are unchanged. And in fact, though initially growing output per person (specifically, following the same reasoning as above, growing by 6\%), output will shrink each period thereafter until the capital stock has shrunk back down to \$500 million.

\subsubsection*{(c)}

Simply giving away resources without investing in improving economic fundamentals can only effect change that would have come about anyway; and in fact, such giveaways intended to grow an economy may quickly be eaten away by depreciation, ultimately involving zero growth relative to before the gift.

\subsection*{5.7}

Recall that steady-state output per capita in Solow is $y^{*} = \bar{A}^{\frac32} \left( \frac{\bar{s}}{\delta} \right)^{\frac12}$. So if $y^{*}$ changes to $\hat{y}^{*}$, the ratio is given by:

\[ \frac{\hat{y}^{*}}{y^{*}} = \left( \frac{\hat{\bar{A}}}{\bar{A}} \right)^{\frac32} \left( \frac{\hat{\bar{s}}}{\bar{s}} \right)^{\frac12} \left( \frac{\delta}{\hat{\delta}} \right)^{\frac12}\]

\subsubsection*{(a)}

$\hat{\bar{s}} = 2\bar{s}$, all else the same. So

\[ \frac{\hat{y}^{*}}{y^{*}} = \sqrt{2} \]

\subsubsection*{(b)}

$\hat{\delta} = 1.1 \delta$, all else the same. So

\[ \frac{\hat{y}^{*}}{y^{*}} = \sqrt{\frac{10}{11}} \]

\subsubsection*{(c)}

$\hat{\bar{A}} = 1.1\bar{A}$, all else the same. So

\[ \frac{\hat{y}^{*}}{y^{*}} = \sqrt{\frac{1331}{1000}} \]

\subsubsection*{(d)}

No parameters change (fundamentals unaffected), so

\[ \frac{\hat{y}^{*}}{y^{*}} = 1 \]

\subsubsection*{(e)}

$\hat{\bar{L}} = 2\bar{L}$, all else the same. But output per person is unaffected by population. So

\[ \frac{\hat{y}^{*}}{y^{*}} = 1 \]

\end{document}