\documentclass{article}

\usepackage{graphicx}
% // For using \textyen to produce the yuan symbol \\
\usepackage[utf8]{inputenc}
\usepackage{newunicodechar}

\newunicodechar{¥}{\textyen}
\DeclareTextCommandDefault{\textyen}{%
  \vphantom{Y}%
  {\ooalign{Y\cr\hidewidth\yenbars\hidewidth\cr}}%
}

\newcommand{\yenbars}{%
  \vbox{
     \hrule height.1ex width.4em
     \kern.15ex
     \hrule height.1ex width.4em
     \kern.3ex
  }%
}
% \\                                               //

\begin{document}

\title{Lecture 2}
\author{Michael Chirico}
\date{\today}

\maketitle

\subsection*{Definition of GDP}
GDP is defined in our text as ``market value of the final goods and services produced in an economy [in a given year]''.

\begin{table}[h]
\centering
\begin{tabular}{|r|r|l|}
\hline
Entity & GDP (\$ USD, Billion) & Equivalent Countries \\
\hline
USA & \$17,000 & - \\
China & \$10,000 & - \\
Japan & \$4,600 & - \\
Germany & \$3,900 & - \\
Tokyo & \$1,900 & Russia, India \\
NYC & \$1,500 & Australia \\
LA & \$800 & Turkey, Saudi Arabia \\
Seoul & \$700 & Switzerland \\
Paris & \$700 &  \\
Chicago & \$600 & Sweden, Argentina \\
Philadelphia & \$400 & Colombia, UAE, Thailand \\
Shanghai & \$400 & \\
Beijing & \$300 & Singapore, Malaysia \\
Xi'an & \$100 & Morocco, Angola \\
\hline
\multicolumn{3}{l}{\scriptsize{All numbers taken from Wikipedia; none are at PPP}}
\end{tabular}
\end{table}

\subsection*{Production = Expenditure = Income}
Modified example from book: production on a durian farm

\subsubsection*{Production}
Count total durian production

\subsubsection*{Expenditure}
Count sales at the roadside durian stand

\subsubsection*{Income}
Count income earned

\section*{Expenditure Approach}

\[
Y = C + I + G + X
\]

Where

\begin{eqnarray*}
Y & = & \mbox{GDP} \\
C & = & \mbox{consumption} \\
I & = & \mbox{investment} \\
G & = & \mbox{government purchases} \\
X & = & \mbox{net exports: exports - imports} \\	
\end{eqnarray*}

Breakdown in the US (textbook):

\begin{figure}[h]
\includegraphics[width=\textwidth]{table_jones_2_1.png}
\end{figure}

Evolution over time (textbook):

\begin{figure}[h]
\includegraphics[width=\textwidth]{figure_jones_2_1.png}
\end{figure}

\subsection*{Breakdown of Exports, China \& USA}

There's an excellent website, The Observatory of Economic Complexity, run by Alex Simoes at MIT, which gives a wealth of data on countries' economies, especially imports/exports: http://atlas.media.mit.edu/en/

In addition to breaking down economies by country, it gives a lot of information about import-export networks (trade flows, which countries are receiving from/giving to which countries, and how much) and even about specific products (so you can understand better where certain raw materials are typically produced and follow their supply chain).

Here is a comparison of imports/exports for China and the US

\newpage

\subsubsection*{Imports}

\begin{figure}[htbp]
\includegraphics[width=\textwidth]{china_import_chart.png}
\end{figure}

\begin{figure}[htbp]
\includegraphics[width=\textwidth]{usa_import_chart.png}
\end{figure}

Can you guess which is China and which is the US?

\newpage

\subsubsection*{Exports}

\begin{figure}[htbp]
\includegraphics[width=\textwidth]{china_export_chart.png}
\end{figure}

\begin{figure}[htbp]
\includegraphics[width=\textwidth]{usa_export_chart.png}
\end{figure}

\section*{Income Approach}

Every dollar paid is a dollar earned.

\begin{figure}[h]
\includegraphics[width=\textwidth]{table_jones_2_2.png}
\end{figure}

About depreciation: if we exclude depreciation, we get the \textit{net domestic product}. We don't really measure it -- it's there to ensure the balance of the different approaches to measuring GDP.

\begin{figure}[h]
\includegraphics[width=\textwidth]{figure_jones_2_3.png}
\end{figure}

\section*{Production Approach}

A hotpot restaurant buys \textyen1000 of chili peppers and oil from one farmer, \textyen1000 of beef from another, \textyen1000 of mushrooms from another, and \textyen1000 of Tsingtao from a beer distributor. They sell all of this for \textyen50000. What is the contribution of GDP of all of this?

If we chose \textyen54000, we'd be double counting. GDP only includes the \textyen50000 sale of \textit{final goods and services} by the restaurant.

An alternative way to think of this is through a \textbf{value-added} approach. Each farmer and the beer distributor contributed \textyen1000 to GDP; the restaurant only added \textyen46000, so that is their contribution.

\section*{Subtleties of GDP Measurement}

GDP does \textit{not} include private services/value-added. There's no way to count your family's herb garden towards GDP unless it is involved in an officially recorded market transaction.

How much does US GDP rise in each of the following scenarios?

\begin{itemize}
\item You spend \$5,000 on tuition
\item You buy a used car from a friend for \$2,500
\item The government spends \$100 million building a dam
\item Two foreign graduate students working as teaching assistants in the US earn \$5,000 each
\end{itemize}



\end{document}