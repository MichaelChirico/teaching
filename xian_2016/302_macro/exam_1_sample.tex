\documentclass{article}

\usepackage{amsmath}
% // For using \textyen to produce the yuan symbol \\
\usepackage[utf8]{inputenc}
\usepackage{newunicodechar}

\DeclareTextCommandDefault{\textyen}{%
  \vphantom{Y}%
  {\ooalign{Y\cr\hidewidth\yenbars\hidewidth\cr}}%
}

\newcommand{\yenbars}{%
  \vbox{
     \hrule height.1ex width.4em
     \kern.15ex
     \hrule height.1ex width.4em
     \kern.3ex
  }%
}
% \\                                               //

\begin{document}

\title{Intermediate Macro Midterm Practice}

\maketitle

\section*{Xian's GDP}

Xian produces Liangpi and Roujiamo. Complete the following table to calculate the size of the Xianese economy.

\begin{table}[htbp]
\centering
\begin{tabular}{|p{5cm}|r|r|r|}
 \hline
 & 2016 & 2017 & Percentage change \\
 \hline
 Quantity of Liangpi & 70 & 75 & \\
 \hline
 Quantity of Roujiamo & 120 & 150 & \\
 \hline
 Price of Liangpi (\textyen) & 25 & 30 & \\
 \hline
 Price of Roujiamo (\textyen) & 8 & 8.4 & \\
 \hline
 Nominal GDP & & & \\
 \hline
 Real GDP in 2016 prices & & & \\
 \hline
 Real GDP in 2017 prices & & & \\
 \hline
 Real GDP in chained prices, benchmarked to 2016 & & & \\
 \hline
\end{tabular}
\end{table}

\section*{Population Growth in the Production Model}

Consider adding simple population growth to the production model. In particular, suppose $\bar{L}_t = (1+g)^t L_0$, where $L_0$ is given as a parameter and represents some initial population. The model is otherwise unchanged and thus consists of five equations:

\begin{itemize}
\item Production: $Y_t = \bar{A} K_t^\frac13 L_t^\frac23$
\item Capital Markets: $r_t = \frac13 \bar{A} \left( \frac{L_t}{K_t} \right)^\frac23$
\item Labor Markets: $w_t = \frac23 \bar{A} \left( \frac{K_t}{L_t} \right)^\frac13$
\item Capital Stock: $K_t = \bar{K}$
\item Labor Stock: $L_t = \bar{L}_t$
\end{itemize}

\begin{enumerate}
\item What is the growth rate of the population in this economy?
\item Solve the model. That is, find equations describing all of the endogenous variables ($w_t$, $r_t$, $K_t$, $L_t$, and $Y_t$). Explain why each variable follows the path that it does.
\item Plot what happens to each of these variables over time. Does this economy reach a steady state?
\item What happens to output per person in this economy? Why?
\item Let's also now grow the capital stock with time, so that equation 4 is replaced with $K_t = \bar{K}_t$, and $\bar{K}_t = (1+G)^t K_0$. Repeat the above for this model. Find a condition which makes output per person constant over time.
\end{enumerate}

\end{document}
