\documentclass{article}

\begin{document}

\title{ECON 301: Intermediate Macroeconomics \\ Summer 2016}

\author{Michael Chirico}

\date{\today}

\maketitle

\section*{Essentials}

\begin{table}[h]
\centering
\begin{tabular}{|l|l|}
\hline
\textbf{Instructor} & Michael Chirico \\
\hline
\textbf{Home Institution} & University of Pennsylvania \\
\hline
\textbf{E-mail} & MichaelChirico4@gmail.com \\
\hline
\textbf{Wechat} & michaelchirico \\
\hline
\textbf{Class Hours} & May 31\textsuperscript{st} - June 30\textsuperscript{th} \\
\hline
\textbf{Office Hours} & Fridays (\textit{by appointment}) \\
\hline
\end{tabular}
\end{table}

\section*{Course Objectives}

\begin{enumerate}
\item Improve understanding of major macroeconomic concepts, such as GDP, inflation and unemployment
\item Acquire analytical skills for examining macroeconomic models
\item Explain the determinants of growth couched in the frameworks of the Solow and Romer models
\item Study macroeconomic fluctuations in a simple, dynamic neoclassical model of the economy
\item Understand basic stylized facts about business cycles, with a particular eye towards comparing \& contrasting model predictions with modern realities
\item Describe the behavior of the economy in the short- and long-run
\item Explain the role of fiscal and monetary policies in the IS-LM model
\item Explain aggregate demand and aggregate supply
\end{enumerate}

\section*{Textbook}

Jones, Charles I. 20013. \textit{Macroeconomics}, 3rd Edition, W. W. Norton \& Company, Inc. ISBN-13: 978-0393923902

\section*{Grading}

\begin{itemize}
\item [15\%] Weekly Homework
\item [15\%] Two In-class Quizzes
\item [30\%] Mid-Term Exam (\textbf{Thursday, June 16})
\item [40\%] Final Exam (\textbf{Thursday, June 30})
\end{itemize}

\section*{Lecture Schedule}

\begin{table}[h]
\centering
\begin{tabular}{c c p{6cm} l}
\textbf{Week} & \textbf{Dates} & \textbf{Topics} & \textbf{Reference} \\
I & 5/31 - 6/3 & Macroeconometrics, Intro to Long-Run Growth & Jones 1-3 \\
II & 6/6 - 6/9 & Solow Growth Model, Technological Innovation & Jones 4-6 \\
III & 6/13 - 6/16 & Macro Labor Markets -- Wages \& Unemployment, Inflation, Short-Run Considerations & Jones 7-9 \\
IV & 6/20 - 6/23 & Introducing the IS-LM Framework and Monetary Policy & Jones 10-13 \\ 
V & 6/27 - 6/30 & Consumption, Investment, Government Intervention, International Finance & Jones 16-20 \\
\end{tabular}
\end{table}

\section*{English-Only Rule}
The reason this course is being taught in English is to benefit the students' mastery of the English language at the same time that they learn powerful tools for economic analysis. English comprehension is a daunting task and even the sharpest learners will continue to struggle with word choice, subtleties of grammar, and colloquial expressions for years to come.

It is very easy and natural (and understandable) to revert to Chinese when speaking with your fellow countrymen, but this is a ruinous choice. Far too many Chinese speakers I've met have been absorbed by the temptation to speak their native tongue when communicating with their neighbors about struggles in class, to their distinct long-term detriment.

As such, there is to be \textbf{absolutely no Chinese (or any other languages/dialects) spoken within the classroom at any time.} \textit{Points will be deducted} as I see fit for violations of this rule. I encourage you to exert the effort and discipline yourself to follow this rule outside of class as well, but I know this is wishful thinking.

\section*{Collaboration}

I believe learning is at its best a collaborative endeavor. I think that working on problem sets as a peer group, if done properly, stands to benefit all parties involved and to deepen understanding, e.g., through paraphrasing and redescribing concepts/thought processes.

As such, I openly encourage students to work together outside of class when working on problem sets. However, such a policy must come with its own proviso to protect it from abuse by unscrupulous participants. 

A small step towards ensuring that all group participants in a work group are contributing, and more importantly, are understanding the work they're submitting, I require each group member to 1) submit their own handwritten version of completed assignments and 2) to cite the names of all group members in the designated portion of each homework. 

This system is easily abused, but just know that as an overall course strategy this is surely dominated by a strategy of working to understand the material. Keep in mind the small portion of your final grade devoted to the homeworks, as compared to the vast majority of your grade which will be derived from in-class work on quizzes and exams (which, intentionally, are to be done individually).

\end{document}