\documentclass{article}

\usepackage{xcolor}

\begin{document}

\title{Intermediate Macro HW 1}

\maketitle

\section{Sinescence of GDP Figures}

Do your best to reproduce Figures 2.1, 2.2, and 2.3 from the textbook, using data for the Chinese economy instead of the US economy. For paralleling Figure 2.1, use as long a timeframe as you are able to find.

\textcolor{red}{Open-ended}

\section{Textbook Exercises, Chapter 2}

\begin{itemize}
\item Jones 2.2

\textcolor{red}{a) \$2 million. (b) \$6,000. (c) \$0 (transfer payments) (d) -\$50 million (imports) (e) \$50 million (exports) (f) \$25,000 - \$100,000 decrease via imports balanced by \$125,000 increase in consumption.}

\item Jones 2.4

\begin{table}[hbtp]
\centering
\caption{Completed table for Jones 2.4}
\begin{tabular}{lrrr}
 & 2016 & 2017 & \% Change, 2016-17 \\
 \hline
 Quantity of oranges & 100 & 105 & \textcolor{red}{5\%} \\
 Quantity of boomerangs & 20 & 22 & \textcolor{red}{10\%} \\
 Price of oranges (\$) & 1 & 1.1 & \textcolor{red}{10\%} \\
 Price of boomerangs & 3 & 3.1 & \textcolor{red}{3.33\%} \\
 Nominal GDP & \textcolor{red}{160} & \textcolor{red}{183.7} & \textcolor{red}{14.8\%} \\
 Read GDP in 2016 prices & \textcolor{red}{160} & \textcolor{red}{171} & \textcolor{red}{6.9\%} \\
 Read GDP in 2017 prices & \textcolor{red}{172} & \textcolor{red}{183.7} & \textcolor{red}{6.8\%} \\
 Read GDP in chained prices, benchmarked to 2017 & \textcolor{red}{171.9} & \textcolor{red}{183.7} & \textcolor{red}{6.8\%} \\
 \hline
\end{tabular}
\end{table}

\item Jones 2.5

\textcolor{red}{Inflation for each index is given by taking the ratio of the GDP deflators for each year.}

\textcolor{red}{Laspeyres: $\frac{\frac{183.7}{171}}{\frac{160}{160}} \approx 7.4\%$}

\textcolor{red}{Paasche: $\frac{\frac{183.7}{183.7}}{\frac{160}{172}} \approx 7.5\%$}

\textcolor{red}{Chain-weighted: $\frac{\frac{183.7}{183.7}}{\frac{160}{171.9}} \approx 7.4\%$}

\item Jones 2.6

\textcolor{red}{%
a) $\frac{\frac{78.9}{45.7}}{14.5} \approx .12$ (b) $\frac{rGDP_I}{rGDP_U} = \frac{\frac{nGDP_I}{P_I}}{\frac{nGDP_U}{P_U}} = \frac{\mbox{nGDP ratio}}{\mbox{price ratio}} \approx .32$ (c) Prices bias our use of GDP as representing production. If India and USA both produce 1,000 gallons of milk, production is equal. But the importance of price to GDP means this won't show up in their nominal GDPs.%
}

\end{itemize}

\section{Textbook Exercises, Chapter 3}

\begin{itemize}

\item Jones 3.1

\textcolor{red}{(a) $700(1.01)^{40}\approx1042$ (b) $700(1.02)^{40}\approx1546$ (c) $700(1.04)^{40}\approx3360$ (d) $700(1.06)^{40}\approx7200$}

\item Jones 3.4

(a); (b)

\begin{table}[htbp]
\centering
\begin{tabular}{lrrrrrr}
\hline
Age & 20 & 25 & 30 & 40 & 50 & 65 \\
\hline
Principal @ 5\% & \$25,000 & \$32,000 & \$41,000 & \$66,000 & \$108,000 & \$225,000 \\
Principal @ 6\% & \$25,000 & \$33,000 & \$45,000 & \$80,000 & \$144,000 & \$344,000 \\
Principal @ 7\% & \$25,000 & \$35,000 & \$49,000 & \$97,000 & \$190,000 & \$525,000 \\
\hline
\end{tabular}
\end{table}

(c); (d)
Simple plot

\item Jones 3.7

\textcolor{red}{Use $\bar{r} = \left(\frac{Y_{2010}}{Y_{1980}}\right)^{\frac1{30}} - 1$}

\item Jones 3.12

\textcolor{red}{This approach ignores compounding of interest.}

\end{itemize}

\end{document}
