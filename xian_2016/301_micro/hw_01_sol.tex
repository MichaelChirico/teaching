\documentclass{article}

\usepackage{xcolor}
% // For using \textyen to produce the yuan symbol \\
\usepackage[utf8]{inputenc}
\usepackage{newunicodechar}

\newunicodechar{¥}{\textyen}
\DeclareTextCommandDefault{\textyen}{%
  \vphantom{Y}%
  {\ooalign{Y\cr\hidewidth\yenbars\hidewidth\cr}}%
}

\newcommand{\yenbars}{%
  \vbox{
     \hrule height.1ex width.4em
     \kern.15ex
     \hrule height.1ex width.4em
     \kern.3ex
  }%
}
% \\                                               //

\begin{document}

\title{Intermediate Micro HW 1}

\maketitle

\section{Nonconvex Indifference Curves}

\begin{enumerate}
\item What does it mean for a set to be convex?

\textcolor{red}{%
Layman's terms: A set is convex if no line drawn between two contained points exits the set.%
}

\textcolor{red}{%
Mathematically: $X$ is convex iff $x,y\in X \Rightarrow \lambda x + (1-\lambda) y \in X \forall \lambda \in [0,1]$%
}

\item What does it mean for indifference curves to be convex?

\textcolor{red}{%
Layman's terms: Convex indifference curves mean a person exhibits diminishing marginal utility, and trades off smoothly among their potential choices.%
}

\textcolor{red}{%
Mathematically: An indifference curve is convex if the upper level set (the set of all higher indifference curves) is convex.%
}

\item Think of a real-life decision problem where indifference curves can reasonably be nonconvex. Explain why you think the ICs are nonconvex in this situation.

\textcolor{red}{Open-ended}

\end{enumerate}

\section{Didi Chuxing and Uber}
Consider the problem of getting around Xi'An over the course of your month-long stay here.

For the purposes of the exercise, you only have two choices for any given trip -- Didi Chuxing (herein Didi) and Uber. 

\begin{enumerate}
\item Argue that Didi and Uber are perfect substitutes. To do so, you'll need to explain what perfect subtitutes are, and why this setting applies to Didi and Uber.

\textcolor{red}{Open-ended}

\item Now argue that Didi and Uber are \textit{not} perfect substitutes. Which description do you think is more accurate? 

\textcolor{red}{Open-ended}

\item Fix your income at \textyen30000 per month. Let the price of an Uber be $p_u$ per mile and that of Didi $p_d$. Assuming perfect substitutes, describe the demand function as a function of the price pair $(p_u, p_d)$.

\textcolor{red}{%
If $p_u > p_d$, $u = 0$ and $d = \frac{30000}{p_d}$. Similarly if $p_d > p_u$. If $p_u = p_d$, we are content with any combination of $u$ and $d$, as long as we exhaust our income.%
}
\end{enumerate}

\section{Textbook Exercises}

\begin{itemize}

\item Varian Exercise 4.3

\textcolor{red}{%
Take the line $y = kx$ through the origin. Take $x < x'$. Then $y = kx < y' = kx'$. Therefore $u(x,y) < u(x', y')$. Thus $(x,y)$ and $(x', y')$ cannot be on the same indifference curve.%
}

\item Varian Exercise 4.4


\textcolor{red}{%
$\sqrt{x_1 + x_2}$ is a monotonic transformation of utility for perfect substitutes.%
}

\textcolor{red}{%
Ditto $13x_1 + 13x_2$.%
}

\item Varian Exercise 5.5

\textcolor{red}{%
This utility function is a monotonic transformation of $u(x_1, x_2) = x_1^{\frac15}x_2^{\frac45}$. That is, it's simply Cobb-Douglas. We know that income shares are determined by the exponents. $\frac45$ will go towards good 2.%
}

\end{itemize}

\end{document}