\documentclass{article}

\usepackage{amsmath}
\DeclareMathOperator*{\Max}{Max}

% // For using \textyen to produce the yuan symbol \\
\DeclareTextCommandDefault{\yuan}{%
  \vphantom{Y}{\ooalign{Y\cr \hidewidth \yuanbars \hidewidth \cr}}%
}

\newcommand{\yuanbars}{%
  \vbox{
     \hrule height.1ex width.4em
     \kern.15ex
     \hrule height.1ex width.4em
     \kern.3ex
  }%
}
% \\                                               //

\usepackage{xcolor}
\newenvironment{solution}{\color{red}}{\color{black}}

\begin{document}

\title{Intermediate Micro In-Class Problems \\ \large Oligopoly \& Game Theory II}

\date{June 29, 2016}

\maketitle

\section*{Profit Sharing Oligopoly}
Two firms are competing in a market. Firm 1 and Firm 2 simultaneously announce quantities, $q_{1}$ and $q_{2}$. The price charged in the market is given by $p = 1- \frac{q_{1}}{2} - \frac{q_{2}}{4}$. Both Firm 1 and Firm 2 have 0 marginal cost of production.

\begin{enumerate}
\item What is Firm 1's reaction function?
\item What is Firm 2's reaction function (note, that it is not the same as Firm 1's)?
\item What is the equilibrium price and equilibrium quantities?
\item Firm 1 and Firm 2 enter into a profit sharing agreement where each receives 25\% of the other firm's profits. Firm 1 and Firm 2 independently decide on $q_1$ and $q_2$.\footnote{One way this can happen is when competitors buy shares in each others' companies.} Given this arrangement, write down each of the two firms' profit functions.
\item What are the equilibrium quantities and price? Are consumers better or worse off as compared with part (3)?
\end{enumerate}

\begin{solution}
\begin{enumerate}
\item[1. \& 2.] $q_{1} = 1 - \frac{q_{2}}{4}$ and $q_{2} = 2 - q_{1}$

To derive the two reaction functions, we begin with the two firms profit functions and then differentiate:
\begin{equation*}
\begin{aligned}
\pi_{1} &= q_{1}(1-\frac{q_{1}}{2}-\frac{q_{2}}{4}) \\
\frac{d \pi_{1}}{d q_{1}} &= 1-q_{1}-\frac{q_{2}}{4} = 0 \\
q^{*}_{1}(q_{2}) &= 1 - \frac{q_{2}}{4} \\
\end{aligned}
\hspace{.5in}
\begin{aligned}
\pi_{2} &= q_{2}(1-\frac{q_{1}}{2} - \frac{q_{2}}{4}) \\
\frac{d \pi_{2}}{d q_{2}} &= 1-\frac{q_{1}}{2}-\frac{q_{2}}{2} = 0 \\
q^{*}_{2}(q_{1})&= 2 - q_{1} 
\end{aligned}
\end{equation*}

\item[3.] $q_{1} = \frac{2}{3}$, $q_{2} = \frac{4}{3}$, and $p=\frac{1}{3}$ 

To solve for the equilibrium, we substitute $q^{*}_{1}(q^{*}_{2}(q_{1}))$ and $q^{*}_{2}(q^{*}_{1}(q_{2}))$:

\begin{equation*}
\begin{aligned}
q_{1} &= 1 - \frac{q_{2}}{4}\\
q_{1} &= 1 - \frac{2 - q_{1}}{4} - c_{1} \\
4q_{1} &= 4-2+q_{1} \\
3q_{1} &= 2 \\
q_{1}^{*} &= \frac{2}{3}
\end{aligned}
\hspace{.5in}
\begin{aligned}
q_{2} &= 2 - q_{1}\\
q_{2} &= 2 - (1 - \frac{q_{2}}{4})  \\
q_{2} &= 1 + \frac{q_{2}}{4} \\ 
\frac{3q_{2}}{4} &= 1\\
q_{2}^{*} &= \frac{4}{3}
\end{aligned}
\end{equation*}

\item[4.] $\pi_{1} = \frac{3q_{1}}{4}(1-\frac{q_{1}}{2}-\frac{q_{2}}{4}) + \frac{q_{2}}{4}(1-\frac{q_{1}}{2} - \frac{q_{2}}{4})$  and 
\\
$\pi_{2} = \frac{q_{1}}{4}(1-\frac{q_{1}}{2}-\frac{q_{2}}{4}) + \frac{3q_{2}}{4}(1-\frac{q_{1}}{2} - \frac{q_{2}}{4} )$
\\

\item[5.] The equilibrium price and quantities are: $p=\frac{16}{37}$, $q_{1} = \frac{12}{37}$, and $q_{2} = \frac{60}{37}$. Consumers are worse off as compared to (3).

To derive the two response functions, we begin with the two firms profit functions and then differentiate:
\begin{equation*}
\begin{aligned}
\pi_{1} =& \frac{3q_{1}}{4}(1-\frac{q_{1}}{2}-\frac{q_{2}}{4}) \\ &+ \frac{q_{2}}{4}(1-\frac{q_{1}}{2} - \frac{q_{2}}{4} ) \\
\frac{d \pi_{1}}{d q_{1}} =& \frac{3}{4}(1-q_{1}-\frac{q_{2}}{4}) - \frac{q_{2}}{8} = 0 \\
=& 1 - q_{1} - \frac{5q_{2}}{12} = 0 \\
q^{*}_{1}(q_{2}) =& 1 - \frac{5q_{2}}{12} \\
\end{aligned}
\hspace{.5in}
\begin{aligned}
\pi_{2} =& \frac{q_{1}}{4}(1-\frac{q_{1}}{2}-\frac{q_{2}}{4}) \\
&+ \frac{3q_{2}}{4}(1-\frac{q_{1}}{2} - \frac{3q_{2}}{4} ) \\
\frac{d \pi_{2}}{d q_{2}} =& \frac{3}{4}(1-\frac{q_{1}}{2}- \frac{q_{2}}{2}) - \frac{q_{1}}{16} = 0\\
=& 2 - q_{1} - q_{2} -  \frac{q_{1}}{6} = 0\\
q^{*}_{2}(q_{1}) =&  \frac{7}{3} - \frac{7q_{1}}{6} \\
\end{aligned}
\end{equation*}

We then substitute into to find the equilibrium:
\begin{equation*}
\begin{aligned}
q_{1} =& 1 - \frac{5q_{2}}{12} \\
q_{1} =& 1 - \frac{5}{12}(2 - \frac{7q_{1}}{6}) \\
q_{1} =& 1 - \frac{5}{6} + \frac{35q_{1}}{72} \\
q_{1} =& \frac{12}{37} \\
\end{aligned}
\hspace{.5in}
\begin{aligned}
q_{2} =& \frac{7}{3} - \frac{7q_{1}}{6} \\
q_{2} =& \frac{7}{3} - \frac{7}{6}(1 - \frac{5q_{2}}{12}) \\
q_{2} =& \frac{7}{6} + \frac{35q_{2}}{72} \\
q_{2} =& \frac{60}{37} \\
\end{aligned}
\end{equation*}

This gives the equilibrium price to be $p = 1 - \frac{12}{2\times 37} - \frac{60}{4 \times 37}=\frac{16}{37}$.

Now, since the price is higher and less is sold, the consumer surplus falls, and we know consumers are worse off as compared to (3).
\end{enumerate}
\end{solution}

\section*{Question 4 (8 points)}
Firm 1 has cost function $C_1 (q) = 3 q + q^2$ where $q$ is the quantity of its output.

Firm 2, a more efficient firm, has a cost function $C_2 (q) = q^2$.

\begin{enumerate}
\item Suppose that each firm can sell all of its output for $p$ per unit. What is the profit-maximizing quantity that Firm 1 will choose (as a function of $p$)? Note that no firm can be compelled to supply if it would lose money.
\item Suppose that each firm can sell all of its output for $p$ per unit. What is the profit-maximizing quantity that Firm 2 will choose (as a function of $p$)? Note that no firm can be compelled to supply if it would lose money.
\item For a given price $p$, what is the total supply for the industry as a whole?
\item Suppose that the market demand is $D = 2-p$. Find the price which clears the market.  How many firms produce at the market clearing price?
\item Now suppose that there is an increase in demand and after the increases, $D = 6.5 - p$. Find the new equilibrium price and quantity sold. Do both firms produce now?
\item What profits do each firm make at the market-clearing price when demand is $D = 6.5 - p$?
\item Suppose a third firm, Firm 3, with the same cost function as Firm 2  is considering entering the market. Only Firm 1 and Firm 2 have a license to operate in this market. Firm 1 is considering selling its license to Firm 3. Suppose Firm 1 can make a take-it-or-leave-it offer to Firm 3, at what price should it offer the license?
\end{enumerate}

\begin{solution}

\begin{enumerate}
\item Both firms maximize profits: $\max_{q \geq 0} p q - C(q)$.\\ \\
For firm 1, the FOC are $p = 3 + 2 q$ which implies that $q = \frac{p-3}{2}$. Notice this term is negative when $p < 3$, which cannot be. Thus, when the per unit price falls below 3, firm 1 will choose not to supply, i.e., choose $q= 0$.
\begin{align*}
q_1^* 
=& \begin{cases} 
\frac{p-3}{2} \text{ if } p \geq 3\\
0 \, \, \, \text{ else}
\end{cases}
\end{align*}

\item For firm 2, the FOC are $p = 2q$ so that $q_2^* = \frac{p}{2}$.\\ \\
At any given price, firm 2 always produces more than firm 1 since it has lower
marginal cost.

\item 
\begin{align*}
S(p) = q_1^* + q_2^* 
=& \begin{cases} 
p - \frac{3}{2} \text{ if } p \geq 3\\
\frac{p}{2} \, \, \, \text{ else}
\end{cases}
\end{align*}

\item With this market demand the equilibrium price cannot be greater than 2. \\ 
Therefore, only firm 2 is going to produce and $S(p) = \frac{p}{2}$.\\ Equalizing demand and supply, we get $2-p = \frac{p}{2}$ $\Rightarrow$ $p^* = \frac{4}{3}$ and the market clearing quantity is  $\frac{2}{3}$.

\item At the new market demand, both firms may be willing to produce. \\
At $p=3$ (the cutoff price for having both firms producing), $D(3) = 3.5 > \frac{3}{2} = S(3)$ and there is excess demand. Therefore, in equilibrium, both firms produce. \\
Equalizing demand and supply, we get $6.5-p =p- \frac{3}{2}$ $\Rightarrow$ $p^* = 4$ and $Q^* =2.5$.

\item Firm 1 is producing $q_1 = \frac{4-3}{2}= 0.5$, while firm 2 is producing $q_2 = \frac{4}{2} = 2$.\\ \\
Profits of firm 1: $\Pi_1 = 4*0.5 - 3*0.5 -0.5^2$ = 0.25 \\
Profits of firm 2: $\Pi_2 = 4*2  -2^2$ = 4 \\

Firms are not making 0 profits because we are in the short run, where no entry can occur so the industry can have positive profits.
\end{enumerate}

\end{solution}

\end{document}