\documentclass{article}

\usepackage{amsmath}
\DeclareMathOperator*{\Max}{Max}

% // For using \textyen to produce the yuan symbol \\
\DeclareTextCommandDefault{\yuan}{%
  \vphantom{Y}{\ooalign{Y\cr \hidewidth \yuanbars \hidewidth \cr}}%
}

\newcommand{\yuanbars}{%
  \vbox{
     \hrule height.1ex width.4em
     \kern.15ex
     \hrule height.1ex width.4em
     \kern.3ex
  }%
}
% \\                                               //

\begin{document}

\title{Intermediate Micro In-Class Problems \\ \large Oligopoly \& Game Theory}

\date{June 28, 2016}

\maketitle

\section*{Triopoly \& Firm Agglomeration}

There are three firms producing Valyrian steel swords. Each firm has constant marginal costs of production of $c$. Each firm $i$ simultaneously chooses a quantity $q_i$. The market price, $p$ per sword is determined by $q_1, q_2, q_3$ using the following inverse demand curve:

\[ p = \Max \{1 - q_1-q_2-q_3, 0\} \]

\begin{enumerate}
\item Fix the quantities chosen by firms 2 and 3 at $q_2$ and $q_3$ respectively. Compute firm 1's profit maximizing quantity choice as a function of $q_2$ and $q_3$. This is firm 1's reaction function.
\item What are the equilibrium quantities chosen by each of the firms?
\item Firms sometimes merge for efficiency reasons. Suppose Firms 1 and 2 were to merge into a single firm. The marginal costs of production of this merged firm will be a constant $0.9c$ per unit. What are the equilibrium quantity choices of both the merged firm and firm 3?
\item Is firm 3 made worse off by the merger of firm 1 and 2?
\item Are consumers made worse off by the merger of firms 1 and 2?
\end{enumerate}

\section*{Price Discrimination}

Devlin-McGregor is the monopoly seller of Provasic to a market of 100 buyers. Buyers are of two types, heavy and light, and there are an equal number of each type. The inverse demand curves are $p_H(q) = 8 - 2q$ for heavy users and $p_L(q) = 2-q$ for light users. Devlin-McGregor produces Provasic at a constant marginal cost of \yuan1 per unit.

\begin{enumerate}
\item If Devlin-McGregor could perfectly discriminate between the two types of buyers and they were restricted to setting a per-unit price, what prices should they charge each type to maximize profit?
\item Suppose the Government were to ban such price discrimination and require Devlin-McGregor to charge all customers the same per-unit price. What price should the firm set to maximize its profit?
\item Does consumer surplus go up or down after the ban?
\item  Suppose the Government allowed Devlin-McGregor to set a single two-part tariff. What would the profit-maximizing two-part tariff be?
\item Suppose the two-part tariff identified in part (4) is in place. The Government is contemplating subsidizing buyers to the tune of \yuan2 per unit of Provasic purchased. Should Devlin-McGregor adjust its two part tariff to account for this and if so, how?
\end{enumerate}

\end{document}