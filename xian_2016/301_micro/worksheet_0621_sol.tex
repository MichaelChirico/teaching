\documentclass{article}

\usepackage{amsmath}

% // For using \textyen to produce the yuan symbol \\
\DeclareTextCommandDefault{\yuan}{%
  \vphantom{Y}{\ooalign{Y\cr \hidewidth \yuanbars \hidewidth \cr}}%
}

\newcommand{\yuanbars}{%
  \vbox{
     \hrule height.1ex width.4em
     \kern.15ex
     \hrule height.1ex width.4em
     \kern.3ex
  }%
}
% \\                                               //

\usepackage{xcolor}
\newenvironment{solution}{\color{red}}{\color{black}}

\begin{document}

\title{Intermediate Micro In-Class Problems \\ \large Monopoly I}

\date{June 21, 2016}

\maketitle

\section*{Yangrou Paomonopoly}
The monopoly supplier of Rangyou Paomo in Xian has production costs of a constant \yuan 2 per unit. For a unit price of $p$, demand for Rangyou Paomo will be $100 - p^2$. 
\begin{enumerate}
\item Write down the monopolist's profit as a function of $p$.
\item Compute the profit-maximizing price she should charge.
\item What is the elasticity of demand at the profit maximizing price? (\textit{Note: if demand is given by $D(p)$, the elasticity of demand is given by the following:})

\[ \varepsilon (p) = - \frac{\partial D}{\partial p} \frac{p}{D(p)} \]

\end{enumerate}

\begin{solution}

\begin{enumerate}
\item Profit is given by:

\[ \Pi(p) = (p - 2)\left( 100-p^2 \right) \]

(i.e., markup times quantity)

\item The first order condition on maximizing profits by choosing $p$ is:

\[ 100 - 3p^2 + 4p = 0 \Rightarrow p^{*} = \frac{4 + \sqrt{1216}}{6} \approx 6.48 \]

\item We know $p$ in the formula; we need an expression for $\frac{\partial D}{\partial p}$ and to find $D(p)$.

\begin{align*}
D(p) &= 100 - \left( p^{*} \right)^2 = \frac{592 - 4\sqrt{304}}{9} \approx 58.03 \\
\frac{\partial D}{\partial p} &= -2p \\
\end{align*}

So we have

\[ \varepsilon(p) = \frac{2p^2}{100 - p^2} \]

\[ \varepsilon(p^{*}) = \frac{154 + 2\sqrt{304}}{148 - \sqrt{304}} \approx 1.45 \]
\end{enumerate}

\end{solution}

\section*{More Monopoly}
Consider a monopolist with production cost function $C(q) = 640 + 20q$, where $q$ is the quantity produced. Let $D(p) = 50 - \frac{p}2$ be the demand-price relationship.

\begin{enumerate}
\item What is the elasticity of demand at the price $p = 20$?
\item At the price $p = 44$, if the monopolist wishes to raise revenue, should he raise or lower the price?
\item What is the monopolist's maximum profit?
\item What is the elasticity of demand at the profit-maximizing price?
\end{enumerate}

\begin{solution}

\begin{enumerate}
\item Using the formula from above:

\[ \varepsilon(20) = - \left( -\frac12 \right) \frac{20}{40} = \frac14 \]

\item At the optimum, the derivative of profit with respect to price is 0; since this function is concave, the direction of the derivative will tell us how to adjust our price.

\[ \Pi(p) = (50 - \frac{p}2) p - 640 - 20(50 - \frac{p}2) = (p - 20)(50 - \frac{p}2) - 640 \]

So that

\[ \Pi'(p) = 60 - p; \]

\[ \Pi'(20) = 40 > 0 \]

So we should raise prices.

\item We quickly see form above that $p^{*} = 60$, from which we quickly deduce:

\[ \Pi(p^{*}) = \frac12 40^2 - 640 = 160 \]

\item Plugging into the above:

\[ \varepsilon(60) = - \left( -\frac12 \right) \frac{60}{20} = \frac32 \]
\end{enumerate}

\end{solution}

\section*{Choosing Price vs. Choosing Quantity}
Consider a monopolist facing a demand curve of the form $D = 50 - 3p$ where $p$ is the unit price. Suppose the monopolist has a constant marginal cost of production of \yuan 3 per unit.

\begin{enumerate}
\item Instead of choosing a unit price $p$ to maximize profit, our monopolist will choose a quantity $q$ to maximize profit. Write down, as function of $q$, the price per unit the monopolist must charge to sell exactly $q$ units. This object is called the \textbf{inverse demand curve}.
\item Write down, as a function of $q$, the monopolist's profit.
\item Write down, as a function of $q$, the monopolist's marginal revenue.
\item Use either the function you identified in part (2) or in part (3) to compute the profit-maximizing quantity. What is it?
\item For your own edification, check that you reach the same conclusion by choosing a profit-maximizing price insted.
\end{enumerate}

\begin{solution}

\begin{enumerate}
\item \[ p = \frac{50 - q}{3} \]

\item \[ \Pi(q) = \frac{50 - q}{3}q - 3q \]

\item \[ MR(q) = \frac{\partial R(q)}{\partial q} = \frac{50 - 2q}{3} \]

\item \[ \Pi(q) = \frac13 (41 - q)q \Rightarrow q^{*} = \frac{41}2 \]

\item \[ \Pi(p) = (p - 3)(50 - 3p) \Rightarrow p^{*} = \frac{59}6 \Rightarrow q^{*} = 50 - \frac{59}2 = \frac{41}2 \]
\end{enumerate}

\end{solution}

\section*{Returns to Scale}

Let $C(q)$ denote the total cost incurred to produce $q$ units of Tsingtao. Decide which of the following cost functions exhibit constant, decreasing and increasing returns to scale.

\begin{enumerate}
\item $C(q) = 5q + 3$ for $q \geq 0$.
\item $C(q) = 2q^2 + 3q + 1$.
\item $C(q) = 5q - q^2$ for $q \leq 5$.
\item $C(q) = 5 q^\frac12$
\end{enumerate}

\begin{solution}
Returns to scale are determined by the sign of the second derivative of the cost function. If $C''(q) < 0$, returns to scale are increasing; if it's positive, the returns are decreasing; and if they're zero, returns are constant.
\begin{enumerate}
\item $C''(q) = 0 \Rightarrow CRS$
\item $C''(q) = 4 > 0 \Rightarrow DRS$
\item $C''(q) = -2 < 0 \Rightarrow IRS$
\item $C''(q) = -\frac54 q^{-\frac32} <0 \Rightarrow IRS$
\end{enumerate}

\end{solution}

\end{document}