\documentclass{article}

\usepackage{xcolor}
\usepackage{graphicx}

\begin{document}

\title{Intermediate Micro Quiz 1}

\maketitle

\section{Shape of Indifference Curves}

Consider the utility function

\[
u(x_1, x_2) = \min\{x_1, x_2^2\}
\]

What do the indifference curves look like for this function?

\color{red}
Indifference curves are the family of functions indexed by all $\bar{u}$ such that

\[
\bar{u} = \min\{x_1, x_2^2\}
\]

If $\min\{x_1, x_2^2\} = \bar{u}$, then either $x_1 = \bar{u}$ and $x_2^2 > x_1$, or $x_2^2 = \bar{u}$ and $x_1 > x_2^2$. An illustration for $\bar{u} = 4$ can be found in Figure \ref{leon}:

\begin{figure}[htbp]
\centering
\includegraphics[width=.5\textwidth]{quiz_01_1_graph}
\caption{Quasi-Leontief Preferences}
\label{leon}
\end{figure}

The story for other values of $\bar{u}$ is similar; thus, like Leontief preferences, all ICs are L-shaped. The difference is that the corner of Leontief preferences follows the 45-degree line; here, they follow the curve $y = \sqrt{x}$.

\color{black}

\section{Demand}

Consider the utility function

\[
u(x_1, x_2) = -\left((x_1 - 2)^2 - (x_2 - 3)^2\right)
\]

With $p_1 = 5$, $p_2 = 2$, and $y = 20$, what bundle of goods will be chosen?

\textit{Hint: What does this function look like? Try to draw its indifference curves}

\color{red}
The key is to recognize that this utility function represents satiated preferences.

The function's global behavior can be inferred from Figure \ref{func}.

We can also see from Figure \ref{ics} that its indifference curves are circles centered at the point $(2, 3)$.

So this individual behaves as follows: If I can afford $(2, 3)$, I buy it, no matter what. Otherwise, I choose the cheapest bundle I can afford that minimizes the distance from $(2, 3)$.

In this case, the cost of the $(2, 3)$ bundle is $16 < 20$, so the demand is $(2, 3)$.

\begin{figure}[htbp]
\centering
\includegraphics[width=.5\textwidth]{quiz_01_2_graph_1.png}
\caption{Globally Satiated Preferences}
\label{func}

\includegraphics[width=.5\textwidth]{quiz_01_2_graph_2.png}
\caption{Circular Indifference Curves}
\label{ics}
\end{figure}

\end{document}