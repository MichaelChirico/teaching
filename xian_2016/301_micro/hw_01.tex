\documentclass{article}

% // For using \textyen to produce the yuan symbol \\
\usepackage[utf8]{inputenc}
\usepackage{newunicodechar}

\newunicodechar{¥}{\textyen}
\DeclareTextCommandDefault{\textyen}{%
  \vphantom{Y}%
  {\ooalign{Y\cr\hidewidth\yenbars\hidewidth\cr}}%
}

\newcommand{\yenbars}{%
  \vbox{
     \hrule height.1ex width.4em
     \kern.15ex
     \hrule height.1ex width.4em
     \kern.3ex
  }%
}
% \\                                               //

\begin{document}

\title{Intermediate Micro HW 1}

\maketitle

\section*{DUE DATE: June 6, 2016 at start of class}

\section{Nonconvex Indifference Curves}

\begin{enumerate}
\item What does it mean for a set to be convex?
\item What does it mean for indifference curves to be convex?
\item Think of a real-life decision problem where indifference curves can reasonably be nonconvex. Explain why you think the ICs are nonconvex in this situation.
\end{enumerate}

\section{Didi Chuxing and Uber}
Consider the problem of getting around Xi'An over the course of your month-long stay here.

For the purposes of the exercise, you only have two choices for any given trip -- Didi Chuxing (herein Didi) and Uber. 

\begin{enumerate}
\item Argue that Didi and Uber are perfect substitutes. To do so, you'll need to explain what perfect subtitutes are, and why this setting applies to Didi and Uber.
\item Now argue that Didi and Uber are \textit{not} perfect substitutes. Which description do you think is more accurate? 
\item Fix your income at \textyen30000 per month. Let the price of an Uber be $p_u$ per mile and that of Didi $p_d$. Assuming perfect substitutes, describe the demand function as a function of the price pair $(p_u, p_d)$.
\end{enumerate}

\section{Textbook Exercises}

Varian Exercises 4.3, 4.4, and 5.5

\end{document}